%%%%%%%%%%%%%%%%%%%%%%%%%%%%%%%%%%%%%%%%%
% University/School Laboratory Report
% LaTeX Template
% Version 3.1 (25/3/14)
%
% This template has been downloaded from:
% http://www.LaTeXTemplates.com
%
% Original author:
% Linux and Unix Users Group at Virginia Tech Wiki 
% (https://vtluug.org/wiki/Example_LaTeX_chem_lab_report)
%
% License:
% CC BY-NC-SA 3.0 (http://creativecommons.org/licenses/by-nc-sa/3.0/)
%
%%%%%%%%%%%%%%%%%%%%%%%%%%%%%%%%%%%%%%%%%

%----------------------------------------------------------------------------------------
%	PACKAGES AND DOCUMENT CONFIGURATIONS
%----------------------------------------------------------------------------------------

\documentclass{article}

\usepackage[version=3]{mhchem} % Package for chemical equation typesetting
\usepackage{siunitx} % Provides the \SI{}{} and \si{} command for typesetting SI units
\usepackage{graphicx} % Required for the inclusion of images
\usepackage{natbib} % Required to change bibliography style to APA
\usepackage{amsmath} % Required for some math elements
\usepackage{tcolorbox} % Provide a frame around text
\usepackage{tabto}
\usepackage[normalem]{ulem}

\usepackage{french}

\setlength\parindent{0pt} % Removes all indentation from paragraphs

\renewcommand{\labelenumi}{\alph{enumi}.} % Make numbering in the enumerate environment by letter rather than number (e.g. section 6)

%\usepackage{times} % Uncomment to use the Times New Roman font

%----------------------------------------------------------------------------------------
%	DOCUMENT INFORMATION
%----------------------------------------------------------------------------------------

\title{BIG DATA \\ TP - usage avanc\'{e} des bases de donn\'{e}es et NoSQL} % Title

\author{Nicolas \textsc{Lagaillardie}} % Author name

\date{\today} % Date for the report

\begin{document}

\maketitle % Insert the title, author and date

%\begin{center}
%\begin{tabular}{l r}
%Date Performed: & January 1, 2012 \\ % Date the experiment was performed
%Partners: & James Smith \\ % Partner names
%& Mary Smith \\
%Instructor: & Professor Smith % Instructor/supervisor
%\end{tabular}
%\end{center}

% If you wish to include an abstract, uncomment the lines below
% \begin{abstract}
% Abstract text
% \end{abstract}

%----------------------------------------------------------------------------------------
%	SECTION 1
%----------------------------------------------------------------------------------------

\section{Objectif}

Suite du TP de gestion de bases de donn\'{e}es : installation et utilisation d'outils de gestion de bases NoSQL.

%\begin{center}\ce{2 Mg + O2 - 2 MgO}\end{center}

% If you have more than one objective, uncomment the below:
%\begin{description}
%\item[First Objective] \hfill \\
%Objective 1 text
%\item[Second Objective] \hfill \\
%Objective 2 text
%\end{description}

\subsection{D\'{e}finitions}
\label{definitions}
\begin{description}
\item[NoSQL]
Ce terme \'{e}signe une famille de syst\`{e}mes de gestion de base de donn\'{e}es (SGBD) qui s'\'{e}carte du paradigme classique des bases relationnelles.
\end{description} 
 
%----------------------------------------------------------------------------------------
%	SECTION 2
%----------------------------------------------------------------------------------------

\section{Installation de Cassandra}

J'ai pr\'{e}f\'{e}r\'{e} installer Cassandra SQL plut\^{o}t que Oracle NoSQL, car plus simple. J'ai suivi la documentation officielle pour l'installation, puis j'ai ex\'{e}cut\'{e} quelques requ\^{e}tes. Voici les commandes entr\'{e}es das le terminal : \\*

\begin{enumerate}
\begin{item}
D\'{e}marrage de Cassandra
\begin{tcolorbox}
\$ sudo service cassandra start
\end{tcolorbox}
\end{item}
\begin{item}
V\'{e}rification du bon d\'{e}marrage
\begin{tcolorbox}
\$ nodetool status
\end{tcolorbox}
\end{item}
\begin{item}
D\'{e}marrage de \textit{cqlsh}
\begin{tcolorbox}
\$ cqlsh
\end{tcolorbox}
\end{item}
\begin{item}
Cr\'{e}ation d'un \textit{keyspace}
\begin{tcolorbox}
\$ CREATE KEYSPACE user WITH REPLICATION = \{ 'class' : 'SimpleStrategy', 'replication\_ factor' : 1 \} ;
\end{tcolorbox}
\end{item}
\begin{item}
Cr\'{e}ation d'un table
\begin{tcolorbox}
\$ use demo; \\*
\$ ...CREATE TABLE eleves ( \\*
\$ ...idEleve text, \\*
\$ ...nom text, \\*
\$ ...prenom text, \\*
\$ ...mail text, \\*
\$ ...telephone text, \\*
\$ ...PRIMARY KEY (idEleve));
\end{tcolorbox}
\end{item}
\begin{item}
Ajouter des colonnes \textit{cours} et \textit{notes}
\begin{tcolorbox}
\$ ALTER TABLE eleves ADD cours list\textless text\textgreater ; \\*
\$ ALTER TABLE eleves ADD notes list\textless int\textgreater ;
\end{tcolorbox}
\end{item}
\begin{item}
Sortir de cqlsh puis lancer Spyder, driver pour Python
\begin{tcolorbox}
\$ EXIT; \\*
\$ Spyder
\end{tcolorbox}
\end{item}
\end{enumerate}

%----------------------------------------------------------------------------------------
%	SECTION 3
%----------------------------------------------------------------------------------------

\section{Peuplement de la table}

J'ai utilis\'{e} Python pour peupler la table avec beaucoup d'informations. Puis, toujours avec Python, j'ai lanc\'{e} des requ\^{e}tes SQL vers Cassandra afin d'afficher des r\'{e}sultats pr\'{e}cis. Vous pouvez observer le code Python dans le fichier \textit{first.py}.

\end{document}